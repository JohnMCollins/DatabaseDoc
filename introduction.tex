\section{Introduction}
\protect\label{section:intro}

This document describes the database and the Python software created to assist
in the analysis of the data from the Red Dots project.

This represents where I got to to the end of May 2023, when I had to withdraw
from the PhD course due to ill-health, note that some parts are incomplete or
``in pieces'' with further development intended.

The document assumes some familiarity with {\py} and also \numpy, {\scipy} and
\mpl. A little familiarity with {\mysql} is helpful, but attempts are made to
explain the functionality where needed. Qt5 is also used to provide GUI-style
software in some places.

\subsection{A note on directories and \texttt{\$PATH} variables}
\protect\label{section:pathvars}

\textbf{Please note that this scheme reflects what I have worked with and
should be easy to change if required, but you may have to change a few
``hardwired'' path names etc in a few places.}

The directory structure, starting from the home directory, used by this, is as
follows.
\vspace{10pt}
\dirtree{%
.1 (Home directory)/.
.2 bin/.
.3 python.
.2 lib/.
.3 dbcred.ini.
.3 perllib/.
.3 pythonlib/.
.2 src/.
.3 astro/.
.4 progs/.
.5 Numpy/.
.6 gsgraphics/.
.6 remfits/.
.6 remfits2/.
.2 .jmc/.
.3 remgeom.
.3 searchparams.
}

The \texttt{Numpy} level is redundant, but has got fossilised and is hardly worth
changing now.

The function of these various directories is as follows:

\begin{description}
\item[bin] Is a directory I always have for holding programs and scripts I like
to work with, not just this project. It is one of the first entries on my
\texttt{\$PATH} variable.
\item[python] is an executable script for invoking the required version of
Python. This is explained later after the section relating to installing \anac,
section \ref{section:pyinvocation}.
\item[lib] Is a directory of reference material not necessarily just for this
project or even for the university.
\item[dbcred.ini] Is a file of credentials for accessing databases (user name
and passwords). It probably wants to be restricted in access.
\item[perllib] Is a library of Perl modules. not necessarily just for this
project or even for the university.
\item[pythonlib] Is a library of {\py} modules. not necessarily just for this
project or even for the university.
\item[src] Contains source modules I am working on, not necessarily all for
this project or even for the university.
\item[astro] Contains material I am working on for the university, for this and
other projects.
\item[progs] Contains programs.
\item[gsgraphics] Contains GUI programs, It should be on your \texttt{\$PATH}.
\item[remfits] Contains shell-level or internal programs. It should be on your
\texttt{\$PATH}.
\item[remfits2] Contains more shell-level or internal programs written later in
the most part. It should be on your \texttt{\$PATH}.
\item[.jmc] Contains configuration data.
\item[remgeom] Contains configuration data for image display (as an XML file).
\item[searchparams] Contains search parameter data (as an XML file).
\end{description}

\subsection{Software requirements}
\protect\label{section:softrequrements}

The software required is:

\begin{enumerate}
  \item \py, including \numpy, {\scipy} and \mpl. This is probably best
  provided by {\anac} as described in Section \ref{section:anaconda} as all the
  necessary libraries such as {\numpy} and {\scipy} will be included
  automatically. Some of the routines in the older versions provided by the
  University machines or standard Linux distributions are not sufficiently
  up-to-date.
  \item {\mysql} or a clone such as \maria. Nothing so esoteric or incompatible
  between the versions is assumed, originally ``views'' were used but these did
  not translate in \maria.\footnote{Although this may have been fixed in a
  later release, I haven't tried as I am not using them now.}
  \item Qt version 5. The software is written to work using Qt version 4, but
  the {\py} ``import'' statements will have to be altered. You will probably need
  ``designer'' as well.
\end{enumerate}

\subsubsection{Installing \anac}
\protect\label{section:anaconda}

\textbf{Please note this section contains some of my own opinions gleaned from
my own experience, some people may disagree with some or all of this, but it is
intended to be helpful.}

First you need to download the latest version of {\anac} from
\texttt{https://www.anaconda.com}, This is changed frequently, so you may want
to revisit it. (It is possible to upgrade but problems can arise if versions
are mixed.)

The place in which {\anac} is installed may depend on what access you have to
your machine. On my own machine, where I have superuser access, I load {\anac} into
\texttt{/usr/local/anaconda3}, so that it is available to all possible users on
the machine without anyone having their own copy. On the University systems,
where I do not have superuser access, I load it into
\texttt{/home/jcollins/lib/anaconda/anaconda3}.
You need to note down the full path name as the installation script does not
understand \texttt{\$HOME} or \~ constructs, much less environment variables
such as \texttt{\$ANACONDA}.

If you are installing to your own machine, you need to type (replacing the name
of the {\anac} downloaded distribution appropriately).

\begin{verbatim}
sudo rm -rf /usr/local/anaconda
sudo bash Anaconda3-2023.03-1-Linux-x86_64.sh
\end{verbatim}

You will be prompted for your password the first time. It is necessary to remove
any previous version first to avoid being interrogated about it.

If you are installing to the University machine, omit the \texttt{sudo} from
those lines and use the correct directory in the first statement.

In either case, be sure to put the full path name of the directory when asked
for the installation location.

After installing \anac, the installation script will ask:

\begin{verbatim}
installation finished.
Do you wish the installer to initialize Anaconda3
by running conda init? [yes|no]
\end{verbatim}

I strongly recommend that you answer ``no'' to this as it will silently fiddle
with your \texttt{.profile}, \texttt{.login} or \texttt{.bashrc} scripts with
possibly confusing results. In any case, if you are installing it as superuser
on your own machine, it would fiddle with the \texttt{.profile} etc files for
\texttt{root} which you certainly do not want.

\subsubsection{Loading required {\py} modules}
\protect\label{section:pymodules}

As well as {\py} itself, you will need a number of modules upon which the
software relies. Some of these are not supplied ``directly'', but via
``channels'', i.e. as extensions of other modules. Modules used in the software
are:

\begin{tabular}{lll}
Module&Description&Dependencies\\\hline
astropy&Astronomy Utilities\\
astroplan&Astronomy Utilities&astropy\\
astroquery&Astronomy Utilities&astropy\\
ephem&Ephemeris\\
jplephem&Additional ephemeris&conda-forge\\
mysql&Database access\\
pymysql&Database access\\
pathtools&Directory scan etc\\
pandoc&Documentation utilities\\
pylint&Scan python code for errors and inconsistencies\\
sshtunnel&SSH access utilities&conda-forge\\\hline
\end{tabular}

\vspace{10pt}
In addition, routines to calculate barycentric dates need to be installed using
\texttt{pip} thus:

\begin{verbatim}
pip install git+https://github.com/shbhuk/barycorrpy.git -U
\end{verbatim}

A shell script \texttt{Lmods} to do all this is provided, possibly the initial
path may need to be edited to set the location to which {\anac} is installed.

\subsubsection{Setting up \$PATH and other environment variables}
\protect\label{section:pathsetting}

This is probably best described by reproducing what I have in my
\texttt{.profile} file as follows, which irrelevant material omitted.
\textbf{Please note that I only use ``bash'' and not ``csh'' and this will have to be rewritten for ``csh''.}

\begin{verbatim}
PYTHONPATH=$HOME/lib/pythonlib
PERLLIB=$HOME/lib/perllib
REMTMP=$HOME/src/astro/tmp
REMLIB=$HOME/src/astro/remfits/lib

for d in $HOME/src/astro/progs/Numpy/remfits{,2} \
		 $HOME/src/astro/progs/Numpy/gsgraphics \
		 $HOME/src/astro/bin \
		 $HOME/src/astro/dlbins \
		 $HOME/bin \
		 do
			if [ -d "$d" ] ; then
    		PATH="$d:$PATH"
    	    fi
		 done
unset d
PATH=:$PATH
export PYTHONPATH PERLLIB REMTMP REMLIB
\end{verbatim}

What this code does is as follows.

Firstly it sets up environment variables pointing to directories used by the
software. Then it sets up (notice that these are in reverse order as the script
pre-pends the directories to the \texttt{\$PATH} variable) the required
directories, followed by the other programs in the \texttt{astro/bin} and \texttt{astro/dlbins} directories.

In my method of working, I like to include the contents of the current directory
in the \texttt{\$PATH}, so I put \texttt{PATH=:\$PATH}. \textbf{You may not
want to do this.}

Finally the environment variables are set up to pass down to other software run
subsequently with the \texttt{export} statement.

\subsubsection{Aliases and program invocation}
\protect\label{section:invocation}

I have the following in my \texttt{.bashrc} file \textit{inter alia}.

\begin{verbatim}
ANACONDA=/usr/local/anaconda3
ANACONDA3=/usr/local/anaconda3
ANACONDALIB=$ANACONDA3/lib/python3.8
ANACONDA3LIB=$ANACONDA3/lib/python3.8
ANACPP=$ANACONDALIB:$ANACONDALIB/site-packages:$PYTHONPATH
ANA3CPP=$ANACONDA3LIB:$ANACONDA3LIB/site-packages:$PYTHONPATH
alias anac='PYTHONPATH=$ANACPP $ANACONDA/bin/python'
alias ianac='PYTHONPATH=$ANACPP $ANACONDA/bin/ipython --pylab'
alias anac3='PYTHONPATH=$ANA3CPP $ANACONDA3/bin/python'
alias ianac3='PYTHONPATH=$ANA3CPP $ANACONDA3/bin/ipython --pylab'
alias pylint="$ANACONDA/bin/pylint"
\end{verbatim}

This enables me to bring up an \textit{ipython} session just by typing:

\begin{verbatim}
ianac
\end{verbatim}

For a while I maintained a version 2 distribution, hence maintaining
\texttt{ianac3}, in case I type that by mistake. If I want the ``plain'' \py, I
just involve \texttt{anac}.

The utility \texttt{pylint} is handy for checking for inconsistencies such as
misspelled variables in files. (You may also want the utility \texttt{notrail}
which cleans up redundant whitespace in files, which \texttt{pylint} moans
about.)

Don't forget you can save a lot of typing of commonly-used stuff by creating
files in \texttt{\~.ipython/profile\_default/startup}, for example in this
directory I have the file \texttt{50-astropy.ipy} containing:

\begin{verbatim}
from astropy.io import fits
from astropy.utils.exceptions import AstropyWarning, AstropyUserWarning
import astroquery.utils as autils
import astropy.units as u
from astropy import coordinates
from astropy.time import Time
import warnings

warnings.simplefilter('ignore', AstropyWarning)
warnings.simplefilter('ignore', AstropyUserWarning)
warnings.simplefilter('ignore', UseWarning)
autils.suppress_vo_warnings()
\end{verbatim}

This loads modules I often want and also turns off warning messages generated
by \texttt{astropy} about obsolescent features in FITS files, which are
included in many of the REM files.

\subsubsection{Python invocation within scripts}
\protect\label{section:pyinvocation}

The {\py} scripts are invoked as programs, using the Linux program invocation
mechanism, in that the first file marked executable of the specified name in the
\texttt{\$PATH} variable is taken.

\begin{enumerate}
  \item If it is a binary, it is directly executed.
  \item If the first line consists of the sequence
  \texttt{\#!~/some/program/name}, then \texttt{/some/program/name} is run with
  the script as first argument followed by the balance of the arguments.
  \item Otherwise the shell is run with the script as the first argument
  followed by the balance of the arguments.
\end{enumerate}

It is normal, therefore, for {\py} scripts to commence with the sequence.

\begin{verbatim}
#! /usr/bin/python
\end{verbatim}

However, if the intention is to run the {\anac} version rather than the
``system'' version of \py, then this will fail. However, it would be a mistake
to use something like

\begin{verbatim}
#! /usr/local/anaconda3/bin/python
\end{verbatim}

as this would beg the question of where {\anac} was installed. (Note that the
\texttt{\#!} construct cannot decode environment variables such as
\texttt{\$ANACONDA}.)

The solution adopted was to put:

\begin{verbatim}
#! /usr/bin/env python3
\end{verbatim}

at the start of each file. The program \texttt{/usr/sbin/env} takes the first
argument, interprets the \texttt{\$PATH} variable and looks for that, then
running the program so found. By putting a shell script \texttt{python3} (and
a similar \texttt{python}) in the user's directory \texttt{\~/bin}, then the
appropriate version of {\py} is selected.

My \texttt{\~/bin/python3} contains:

\begin{verbatim}
#! /bin/sh

exec /usr/local/anaconda3/bin/python "$@"
\end{verbatim}

But in principle it could be set to choose the version of {\py} to run according
to the context.

\subsection{MySQL}
\protect\label{section:mysql}

The usage of {\mysql} avoids constructs different or absent between the versions
of {\mysql} and \maria, in particular ``views''. No particular ISAM engine is
specified.

\subsubsection{Storage of FITS files}
\protect\label{section:fitsstorage}

I decided to store the actual FITS files in use in the database itself rather
than having a separate file store. The reasons for doing this are that.

\begin{enumerate}
  \item The database store is and many systems separate and differently
  controlled from the file system.
  \item It is possible to manage separately the FITS files from from other
  database rows. (This is described later in section \ref{section:indices}.)
  \item Utilities such as \textit{locate}, \textit{slocate} and \textit{mlocate}
  are rendered less useful if clogged by many similarly-named files.
\end{enumerate}

(It doesn't seem to be clearly documented anywhere, but it would have saved me
some time if I'd known, so I include the following.)
This may raise a problem with some versions of Linux which may restrict the locations of the data store used by \mysql. For
example, on my computer, running  \textit{Ubuntu} Linux, I have a 3TB disk
mounted on \texttt{/home2} for the purpose of holding {\mysql} files (under \texttt{/home2/mysql}).

In order to provide for this, the data directory for {\mysql} is selected by
inserting or adjusting the setting for the this in the file
\texttt{/etc/mysql/mysql.conf.d/mysqld.cnf} to the following.

\begin{verbatim}
datadir	= /home2/mysql
\end{verbatim}

However this did not work at first, because there is a security module called
\textit{apparmor} on some Linux systems including \textit{Ubuntu} which
restricts the areas on the filesystem which a given subsystem can access,
including via a symbolic link, as it operates by physical location. It is
necessary to insert the following file as \texttt{/etc/apparmor.d/local/usr.sbin.mysqld}.

\begin{verbatim}
/home2/mysql/ r,
/home2/mysql/** rwk
\end{verbatim}

Various systems will probably need to be restarted, not just \mysql, after
doing this, possibly it is easiest to reboot.

\subsubsection{Passwords for and access to \mysql}
\protect\label{section:mysqlpw}

I wanted to have a uniform access to the {\mysql} data regardless of whether I
was using the shell tool, {\py} or Perl and whether the database being accessed
was held locally, on a different system with direct network access, or via an
SSH tunnel.

I achieved this by creating a {\py} module \textit{dbops} with a simple
interface:

\begin{verbatim}
import dbops
......
dbase = dbops.opendb('databasename')
\end{verbatim}

This returns a database connection to \texttt{databasename} after ``doing
whatever is needed'' to achieve the connection.

``Whatever is needed'' is set up in a file which can be one or more of:

\begin{enumerate}
  \item The file \texttt{/etc/dbcred.ini}.
  \item The file \texttt{\$HOME/lib/dbcred.ini}, i.e. the file
  \texttt{dbcred.ini} in \texttt{lib} directory in the user's home directory.
  Probably this is the recommended version.
  \item The file \texttt{.dbcred.ini} in the current directory when the program
  is invoked.
\end{enumerate}

If there are conflicting entries in the various files, the files are read in
that order so the last read entry will apply.

\textbf{{\bomb} Note that this isn't very secure as passwords are stored in
plain text but the assumption is made that the database contents are not highly secret and
shared in an academic community who don't want to spend their time typing
passwords. It is best not to use this for access to databases with sensitive
information. Perhaps make sure that the files are not ``world readable'' at
least.}

The format of these files is in ``configuration file'' format and the contents
are probably best illustrated by actual examples.

I set up two versions of the database, one on the cluster, which I called
\texttt{cluster} and one on my home machine, which I called \texttt{remfits}. I
also needed to access the database for the red dots project, which I called
\texttt{rdots}.

The cluster database can only be accessed locally and an SSH tunnel has to be
used to access it from outside. The other two databases can be accessed from
outside.

On the cluster, the file \texttt{\~/lib/dbcred.ini} contains the
following.\footnote{Of course the passwords have been changed from the real
ones.}

\begin{verbatim}
[DEFAULT]
database=mydb
host=localhost
password=mypass
remoteport=3306
user=jmc

[cluster]
database=jcollins
host=localhost
password=ABCDEFGHIJ
user=jcollins

[rdots]
database=REMImgs
host=ross.iasfbo.inaf.it
password=ANCDEFGHIJ
user=rdots

[remfits]
database=remfits
host=brute.jmc.me.uk
password=ABCDEFGHIJ
user=remupd
\end{verbatim}

On my home machine, the file \texttt{\~/lib/dbcred.ini} contains the following:

\begin{verbatim}
[DEFAULT]
database=mydb
host=localhost
password=mypass
remoteport=3306
user=jmc

[cluster]
database=jcollins
host=stri-cluster.herts.ac.uk
localport=9999
login=jcollins
password=ABCDEFGHIJ
user=jcollins

[rdots]
database=REMImgs
host=ross.iasfbo.inaf.it
password=ABCDEFGHIJ
user=rdots

[remfits]
database=remfits
host=localhost
password=ABCDEFGHIJ
user=remupd
\end{verbatim}

This illustrates the various cases. Where the database can be accessed directly,
either on the local host or directly over the internet, 4 fields need to be
defined. The \texttt{DEFAULT} section provides defaults if fields are omitted,
the only important one is \texttt{remoteport}.

\begin{description}
\item[host] gives the host name (or IP address) which is \texttt{localhost}
where it is the local machine.
\item[database] gives the database name.
\item[user] gives the user name on {\mysql} under which access is performed,
which should not be confused with a Linux user name.
\item[password] gives the {\mysql} password.
\end{description}

Access to \texttt{remfits} and \texttt{rdots} are examples of this in both
cases, as is access to \texttt{cluster} on the cluster
itself.\footnote{Possibly confusingly three fields are all \texttt{jcollins} on
the cluster.}

For SSH tunnels, three additional fields must also be supplied. These are

\begin{description}
\item[login] gives the Linux login name to log into the remote SSH server.
\item[localport] gives the port number to allocate on the local server to access
the SSH tunnel.
\item[remoteport] gives the port number on the remote machine to provide a
tunnel for. This is specified in the \texttt{DEFAULT} section as it is always 3306
for \mysql.
\end{description}

Please note that I set up a key access system for SSH rather than passwords. In
fact for security I disable SSH access from outside using passwords.
