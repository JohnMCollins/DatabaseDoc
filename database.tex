\section{Database structure}
\protect\label{section:databasestruct}

In this section the layout of the database is described.

\subsection{Indices}
\protect\label{section:indices}

I created a system of ``indices'' to quickly refer to:

\begin{enumerate}
  \item Observations, as \texttt{obsind},
  \item Objects, as \texttt{objind}.
  \item FITS files, usually as just \texttt{ind}.
  \item Find results, usually as just \texttt{frind}.
\end{enumerate}

I had planned to extend this to processed image files, stored as a {\numpy} 3D
array, with the first plane giving values and the second plane giving calculated
standard deviations.

The indices are stored as {\mysql} unsigned integers, with the field in the
corresponding record ab auto-increment primary key. With apologies to those
familiar with \mysql, this means that when a new record is created in the
relevant table, unless the key value is supplied, which it is probably an error
to do, a new unique value for the key is created and inserted into the record.
This value can be fetched immediately afterwards within {\py} from the database
``cursor'' from the field \texttt{lastrowid}, thus:

\begin{verbatim}
cursor.execute{"INSERT INTO .....")
ind = cursor.lastrowid
\end{verbatim}

All of the library routines do this where appropriate.

The reason for doing this is to give a fast reference to and database access of
a record which has already been selected using other more complicated criteria,
for example an observation with a particular date and time and filter do not have
to be selected each time using those criteria, but only with the relevant index
number.

This is also useful for generating composite tables, for example if you wanted
to list filter names, dates and aducounts for all the ADU counts currently
calculated\footnote{This is probably not very sensible, but is only for a
simple example.} you could use the statement\footnote{{\mysql} keywords are
case-insensitive but conventionally I put them in upper case for improved
readability.}:

\begin{verbatim}
SELECT filter,date_obs,aducount FROM obsinf INNER JOIN aducalc ON obsinf.obsind=aducalc.obsind
\end{verbatim}

This produces a combined table out of the selected fields of each table (you
might have to qualify them with the table name in places as with the fields in
the \texttt{ON} clause) where the given indices match.

{\mysql} provides for ``views'' which predefine common cases of such
\texttt{JOIN} operations but these are not compatible or portable between
different versions of that and also \maria.

\subsubsection{FITS files}
\protect\label{section:fitsindex}

A particularly important case of this is for storing FITS files in a
table \texttt{fitsfile} separately from the observation data table
\texttt{obsinf}.

I did this because I wanted to store the FITS files in the database to avoid
having a plethora of files with similar names on the disk. However manipulating
the rows in a table containing large amounts of data can be very processor and
disk intensive, especially if fields are being changed in the tables, so it
makes sense to have minimal data alongside the FITS files apart from the files
themselves, with the statistics put in the observation data table.

So in the table \texttt{fitsfile} is stored \texttt{ind} which is an
auto-increment unsigned int primary key, and the column \texttt{fitsgz}, which
is a ``longblob'' type.

In the table \texttt{obsinf} is stored the corresponding \texttt{ind}, not to be
confused with \texttt{obsind} which is the index of the observation record. The
\texttt{ind} value is used to look up the record in the \texttt{fitsfile} table
to fetch the actual FITS file.

If the FITS file for the observation is not stored in the database, then the
value of \texttt{ind} will be zero. This will be the case for objects not in the
Red Dots Project target list, for filters other than the \texttt{g}, \texttt{r},
\texttt{i} and \texttt{z} and for missing or unusable images.

The library routines are set up to transparently obtain the FITS file from the
database if it is stored there, or to download the file directly if possible.

Arrays of daily bias and flat files and standard master flat and bias files are
also stored here.

\subsubsection{{\numpy} arrays}
\protect\label{section:numpyarrays}

I had intended to move the storage of {\numpy} arrays to the database, for
holding either arrays derived from images as 2 planes, with one plane of pixel
values and the other of calculated standard error on a per-pixel basis, or bad
pixel arrays.

I had set up the table \texttt{stdarray} to this end but had not used it. The
usage was similar to that of \texttt{fitsfile}.

\subsection{Tables and usage in database}
\protect\label{section:databasetabs}

The following sections explain each of the various tables in the database and
their various fields.

\subsubsection{\texttt{obsinf} table}
\protect\label{section:obsinf}

The following fields are provided in the \texttt{obsinf} table.

\begin{description}
\item[\tt ind] is an integer giving the index of the FITS file data in the
\texttt{fitsfile} table, or zero if there is none.
\item[\tt serial] is the serial number of the observation in the Red Dots
database.
\item[\tt radeg] is the Right Ascension of the object in degrees.
\item[\tt decdeg] is the Declination of the object in degrees.
\item[\tt object] is the object name of the target as specified in the Red Dots
database, not necessary the same as the object name used in other tables.
\item[\tt dithID] is the dither pattern number, applicable to REMIR data and zero
for the visible light filters.
\item[\tt filter] is the filter name.
\item[\tt date\_obs] is the observation date and time.
\item[\tt mjdobs] is the observation date in modified Julian date format.
\item[\tt bjdobs] is the same corrected for barycentric date and time.
\item[\tt exptime] is the exposure time in seconds.
\item[\tt gain] is the gain in the observation.
\item[\tt seeing] is the seeing value.
\item[\tt airmass] is the airmass value.
\item[\tt moonphase] is the phase of the moon (between 0 and 1 if defined)
\item[\tt moondist] is the distance of the moon from the target in degrees, up to
180.
\item[\tt quality] is a quality value, between 0 and 1 for the observation. Zero is
completely unacceptable and 1 is maximum acceptability.
\item[\tt rejreason] is a brief textual description of why the observation is
unacceptable if it is to be ignored.
\item[\tt \tt fname] is a short file name of the observation in the Red Dots
project files.
\item[\tt \tt ffname] is a full file name of the observation in the Red Dots project
files.
\item[\tt obsind] is the index of the observation, which is a unique unsigned
primary key with auto-incremented value. It is used elsewhere to quickly look up
the observation.
\item[\tt orient] is a rough orientation of the image, from 0 to 3, with 0
representing North at the top, 1 North on the left, 2 North at the bottom and 3
North on the right.
\item[\tt nrows] is the number of effective rows in the image.
\item[\tt ncols] is the number of effective columns in the image.
\item[\tt startx] is the starting column number on the CCD (bottom left) of the
image. Columns are numbered from zero.
\item[\tt starty] is the starting row number on the CCD (bottom left) of the image.
Rows are numbered from zero.
\item[\tt rowoffset] gives the possible fractional pixel row number by which the
target centre differs from the position which it is supposed to be, expressed as
where it is on the image minus where it is expected to be.
\item[\tt coloffset] gives the possible fractional pixel column number by which the
target centre differs from the position which it is supposed to be, expressed as
where it is on the image minus where it is expected to be.
\item[\tt minv] gives the minimum pixel value of the image.
\item[\tt maxv] gives the maximum pixel value of the image.
\item[\tt sidet] gives the number of pixels trimmed from each side of each image
before calculating the mean and other statistics of the array.
\item[\tt median] gives the median pixel value of the array.
\item[\tt mean] gives the mean pixel value of the array.
\item[\tt std] gives the standard deviation of the array.
\item[\tt skew] gives the skewness of the array.
\item[\tt kurt] gives the kurtosis of the array.
\end{description}

The last two fields are only included for consistency with the
\texttt{iforbinf} tables which in turn are only maintained because they are
used in the compilation of the Red Dots master flat files.

Please see the documentation on the \textbf{listobs.py} utility in section
\ref{section:listobs} for information on further using these fields for
selection of images.

\subsubsection{\texttt{iforbinf} table}
\protect\label{section:iforbinf}

The \texttt{iforbinf} table is used to store information about individual daily
bias and flat images, with the information itself held in the \texttt{fitsfile}
tan;e/

\begin{description}
\item[\tt ind] is an integer giving the index of the FITS file data in the
\texttt{fitsfile} table, or zero if there is none.
\item[\tt serial] is the serial number of the file in the Red Dots database.
\item[\tt filter] is one of \texttt{g}, \texttt{i}, \texttt{r} and \texttt{z}
and gives the filter in question. (They are only provided for these filters.)
\item[\tt typ] (correctly spelt without an ``e'' to avoid clashes with {\mysql}
keyword) gives the type of file in question \texttt{bias} or \texttt{flat}.
There is provision for a type \texttt{dark} as these were originally also
provided but none during the observation periods.
\item[\tt date\_obs] gives the date and time of the image.
\item[\tt mjdobs] gives the modified Julian date of the image.
\item[\tt exptime] gives the exposure time of the image in seconds.
\item[\tt gain] gives the gain setting.
\item[\tt rejreason] is either null or gives a reason for rejecting the image.
\item[\tt fname] is a short file name of the observation in the Red Dots
project files.
\item[\tt ffname] is a full file name of the observation in the Red Dots project
files.
\item[\tt iforbind] is the index of the file, which is a unique unsigned
primary key with auto-incremented value. It is used elsewhere to quickly look up
the file.
\item[\tt nrows] is the number of effective rows in the image.
\item[\tt ncols] is the number of effective columns in the image.
\item[\tt startx] is the starting column number on the CCD (bottom left) of the
image. Columns are numbered from zero.
\item[\tt starty] is the starting row number on the CCD (bottom left) of the
image. Rows are numbered from zero.
\item[\tt minv] gives the minimum pixel value of the image.
\item[\tt maxv] gives the maximum pixel value of the image.
\item[\tt sidet] gives the number of pixels trimmed from each side of each image
before calculating the mean and other statistics of the array.
\item[\tt median] gives the median pixel value of the array.
\item[\tt mean] gives the mean pixel value of the array.
\item[\tt std] gives the standard deviation of the array.
\item[\tt skew] gives the skewness of the array.
\item[\tt kurt] gives the kurtosis of the array.
\end{description}

The last two fields are only included because they are used in the compilation of the Red Dots
master flat files.\footnote{Although I believe that they are not calculated
correctly in those files.}

Please see the documentation on the \textbf{listiforb.py} utility in section
\ref{section:listiforb} for information on further using these fields for
selection of bias or flat files.

\subsubsection{\texttt{forbinf} table}
\protect\label{section:forbinf}

The \texttt{forbinf} table is used to store details of the master flat and
bias files provided by the Red Dots project with the actual FITS files stored in
\texttt{fitsfile} and contains the following fields:

\begin{description}
\item[\tt year] Year for which stored, e.g. 2017, 2022.
\item[\tt month] Month for which stores as 1 to 12.
\item[\tt filter] is one of \texttt{g}, \texttt{i}, \texttt{r} and \texttt{z}
and gives the filter in question. (They are only provided for these filters.)
\item[\tt typ] (correctly spelt without an ``e'' to avoid clashes with {\mysql}
keyword) gives the type of file in question \texttt{bias} or \texttt{flat}.
\item[\tt gain] gives the gain setting.
\item[\tt nrows] is the number of effective rows in the image.
\item[\tt ncols] is the number of effective columns in the image.
\item[\tt startx] is the starting column number on the CCD (bottom left) of the
image. Columns are numbered from zero.
\item[\tt starty] is the starting row number on the CCD (bottom left) of the
image. Rows are numbered from zero.
\item[\tt rejreason] is either null or gives a reason for rejecting the image.
\item[\tt fitsind] (not \texttt{ind}, this is probably an inconsistency but
not worth changing) is an integer giving the index of the FITS file data in the
\texttt{fitsfile} table, or zero if there is none.
\end{description}

These files are not used ``in anger'' as I do not agree with the ways in which
they are calculated, but are kept for reference.

\subsubsection{\texttt{fitsfile} table}
\protect\label{section:fitsfile}

The \texttt{fitsfile} table is used to store FITS files for the \texttt{obsinf},
\texttt{iforbinf} and \texttt{forbinf} tables and contains the following fields:

\begin{description}
\item[\tt ind] is the unsigned int used as an auto-incremented primary key to
refer to each entries.
\item[\tt side] gives the dimensions of the array and is either 512, indicating
$512 \times 512$ or 1024 indicating $1024 \times 1024$.
\item[\tt nrows] gives the number of rows in the array after trimming from the
top any rows of zero elements.
\item[\tt ncols] gives the number of columns in the array after trimming from
the right any columns of zero elements.
\item[\tt startx] gives the starting column number on the CCD (bottom left) of the
image. Columns are numbered from zero.
\item[\tt starty] is the starting row number on the CCD (bottom left) of the
image. Rows are numbered from zero.
\item[\tt fitsgz] is a ``longblob'' type giving the actual FITS file.
\end{description}

\subsubsection{\texttt{stdarray} table}
\protect\label{section:stdarray}

The \texttt{stdarray} table is used to store {\numpy} array files and contains
the following fields:

\begin{description}
\item[\tt ind] is the unsigned int used as an auto-incremented primary key to
refer to each entries.
\item[\tt arrgz] is a ``longblob'' type giving the actual {\numpy} array file.
\end{description}

This table is not in use but is intended to hold processed image files after
calculating the standard deviation for each pixel. It is also intended to hold
bad pixel files. It was intended to move all these files to the database rather
than as a huge collection on disk.

\subsubsection{\texttt{objdata} table}
\protect\label{section:obsdata}

The following fields are provided in the \texttt{objdata} table to store details
of objects.

\begin{description}
\item[\tt objname] is the name of the object used internally. Where several
names are available, the shortest one is used, for example {\ross} is stored
here as \texttt{GJ729}. (Other names are stored in the \texttt{objalias} table.)
\item[\tt ind] is the unsigned integer auto-increment primary key used to
identify the object elsewhere.
\item[\tt objtype] is the type of the object, where known.
\item[\tt dispname] is the ``display name'' of the object for when it is
displayed as a text string, for example {\ross} is given as \texttt{Ross 154}.
\item[\tt latexname] is the ``Latex name'' of the object, on the assumption that
key objects such as \prox, {\bstar} and {\ross} are set up as the macros
\texttt{\\prox}, \texttt{\\bstar} and \texttt{\\ross} respectively for inclusion
of auto-generated tables and similar directly into a report.
\item[\tt vicinity] gives the ``vicinity'' of the object for listing of possible
objects close to one of the Red Dots project targets of \prox, {\bstar} and
{\ross} respectively. These use the the same names as the \texttt{objname}
field, for example if the object is in the vicinity of {\ross} then this field
will contain \texttt{GJ729}.
\item[\tt label] is a label to be used when images are displayed concentrating
on a given target vicinity.
The objects are sorted into order in of descending frequency in which they appear in images,
then into descending overall brightness. After assigning all the targets the
label \texttt{A}, labels are assigned \texttt{B}, \texttt{C} etc through to
\texttt{Z} then \texttt{A1} through to \texttt{Z1}, then \texttt{A2} through to
\texttt{Z2} and so forth.
\item[\tt dist] is the distance in light years, where known.
\item[\tt rv] is the radial velocity in km/s. A negative value indicates that it
is approaching.
\item[\tt radeg] is the Right Ascension in degrees. (0 to 360).
\item[\tt rapm] is the proper motion of the Right Ascension of the object in
milli-arcseconds/year.
\item[\tt decdeg] is the Declination of the object in degrees (-90 to 90)
\item[\tt decpm] is the proper motion of the Declination of the object in
milli-arcseconds/year.
\item[\tt gmag] gives the \texttt{g} filter magnitude of the object.
\item[\tt imag] gives the \texttt{i} filter magnitude of the object.
\item[\tt rmag] gives the \texttt{r} filter magnitude of the object.
\item[\tt zmag] gives the \texttt{z} filter magnitude of the object.
\item[\tt Hmag] gives the \texttt{H} filter magnitude of the object (this is
currently unused by the software, but included for later processing of REMIR
infrared images).
\item[\tt Jmag] gives the \texttt{J} filter magnitude of the object (this is
currently unused by the software, but included for later processing of REMIR
infrared images).
\item[\tt Kmag] gives the \texttt{K} filter magnitude of the object (this is
currently unused by the software, but included for later processing of REMIR
infrared images).
\item[\tt gbri] gives the weighted average flux for images with the \texttt{g}
filter.
\item[\tt gbrisd] gives the standard deviation of flux for images with the
\texttt{g} filter.
\item[\tt ibri] gives the weighted average flux for images with the \texttt{i}
filter.
\item[\tt ibrisd] gives the standard deviation of flux for images with the
\texttt{i} filter.
\item[\tt rbri] gives the weighted average flux for images with the \texttt{r}
filter.
\item[\tt rbrisd] gives the standard deviation of flux for images with the
\texttt{r} filter.
\item[\tt zbri] gives the weighted average flux for images with the \texttt{z}
filter.
\item[\tt zbrisd] gives the standard deviation of flux for images with the
\texttt{z} filter.
\item[\tt Hbri] gives the weighted average flux for images with the \texttt{H}
filter.
\item[\tt Hbrisd] gives the standard deviation of flux for images with the
\texttt{H} filter.
\item[\tt Jbri] gives the weighted average flux for images with the \texttt{J}
filter.
\item[\tt Jbrisd] gives the standard deviation of flux for images with the
\texttt{J} filter.
\item[\tt Kbri] gives the weighted average flux for images with the \texttt{K}
filter.
\item[\tt Kbrisd] gives the standard deviation of flux for images with the
\texttt{K} filter.
\item[\tt apsize] gives the currently-calculated optimal aperture radius in
pixels.
\item[\tt irapsize] gives the currently-calculated optimal aperture radius for
infra-red filters in pixels.
\item[\tt apstd] gives the standard deviation of the currently-calculated
optimal aperture radius.
\item[\tt irapstd] gives the standard deviation of the currently-calculated
optimal aperture radius for infrared filters.
\item[\tt basedon] gives the number of observations that \texttt{apsize} is
based upon.
\item[\tt irbasedon] gives the number of observations that \texttt{irapsize} is
based on.
\item[\tt variability] gives the variability of the object as assessed by the
software from 0 (none) to 1 (max).
\item[\tt suppress] indicates that the object should for one reason or another
be ignored in searches.
\item[\tt invented] indicates that the name given is ``invented'' as the object
does not appear in any catalogue currently scanned.
\item[\tt usable] indicates that the object is potentially usable as a reference
object.
\end{description}

There is some overlap between the meanings of \texttt{variability},
\texttt{suppress} and \texttt{usable} which is documented where appropriate. The
field \texttt{variability} is more flexible as it is on a sliding scale whereas
the others are binary.

\subsubsection{\texttt{objalias} table}
\protect\label{section:objalias}

The following fields are provided in the \texttt{objalias} table, to store
details of alternative object names.

\begin{description}
\item[\tt objname] is the object name for which an alias is being given. This
corresponds to \texttt{objname} in the \texttt{objdata} table, i.e. it is the
shortest name (available when the tables were set up), for example {\ross} is stored
here as \texttt{GJ729}, but an alias will be given as a row in this table,
specifying {\ross} as an alias.
\item[\tt alias] is the alias name. Note this is also a key to supply
\item[\tt source] gives the source of the alias, for example ``Simbad'' or
``Manually inserted''.
\item[\tt sbok] indicates whether the alias is given in Simbad.
\end{description}

Most of the library routines are set up to transparently fetch the data for an
object from the \texttt{objdata} table from whatever alias they are invoked by,
by reference to the \texttt{objalias} table where necessary.

\subsubsection{\texttt{objpm} table}
\protect\label{section:objpm}

The following fields are provided in the \texttt{objpm} table, used to keep a
cache of proper motions so that they do not have to be continually recalculated.

\begin{description}
\item[\tt objind] gives the index of the object, as set up in the
\texttt{objind} field of the \texttt{objdata} table.
\item[\tt obsdate] gives the date upon which the item is based. (Not bothering
with time).
\item[\tt radeg] gives the Right Ascension in degrees.
\item[\tt decdeg] gives the Declination in degrees.
\end{description}


\subsubsection{\texttt{objedit} table}
\protect\label{section:objedit}

The following fields are provided in the \texttt{objedit} table. This is used
for storing pending changes to objects for performing together.

\begin{description}
\item[\tt ind] is an auto-increment unsigned integer primary key for referring
to the operation.
\item[\tt op] is one of the operations defined below.
\item[\tt objind] is the index of the object being referred to.
\item[\tt apsize] is an aperture size where relevant.
\item[\tt nrow] is where relevant a row number in an image.
\item[\tt ncol] is where relevant a column number in an image.
\item[\tt radeg] is where relevant a Right Ascension value in an image.
\item[\tt decdeg] is where relevant a Declination value in an image.
\item[\tt done] indicates that the edit has been done.
\item[\tt objname] is where relevant an object name.
\item[\tt dispname] is where relevant a display name.
\item[\tt latexname] is where relevant a Latex name.
\item[\tt obsfile] is where relevant a full path to a file name for an image.
\end{description}

The \texttt{op} filed is one of:

\begin{description}
\item[\tt NEW] indicates that a new object (\texttt{objdata} entry) is to be
created and the aperture calculated from the image supplied in
\texttt{obsfile}. The coordinates are in the region delineated by \texttt{nrow}
and \texttt{ncol}.
\item[\tt NEWAP] indicates that a new object (\texttt{objdata} entry) is to be
created and the aperture specified in \texttt{apsize}. The coordinates are in
the region delineated by \texttt{nrow} and \texttt{ncol}.
\item[\tt HIDE] indicates that the object specified is to be marked ``hidden''
in the image.
\item[\tt DELDISP] indicates that the display and Latex names are to be cleared
and reset to the same as the \texttt{objname} field for the object.
\item[\tt NEWDISP] indicates that the display and Latex names are to be set as
specified.
\item[\tt SETAP] indicates that the aperture size is to be set to that given by
\texttt{apsize}.
\item[\tt CALCAP] indicates that the aperture size is to be calculated from the
image given in \texttt{obsfile}.
\end{description}

This probably wants revision as files are moved to the database.

\subsubsection{\texttt{findresult} table}
\protect\label{section:findresult}

The following fields are provided in the \texttt{findresult} table, which is
used to store details of search results.

\begin{description}
\item[\tt obsind] gives the observation index.
\item[\tt objind] gives the object index.
\item[\tt ind] is the table's own index for find results.
\item[\tt nrow] gives the row number of the centre of the object (this actually
might be fractional if the centre lies between pixels.)
\item[\tt ncol] gives the column number of the centre of the object (this
actually might be fractional if the centre lies between pixels.)
\item[\tt radeg] gives the Right Ascension value in degrees.
\item[\tt decdeg] gives the Declination value in degrees.
\item[\tt rdiff] gives the difference from ``expected'' row number, i.e. where
it actually is in the image minus what it is expected to be (this is ``on top''
of \texttt{rowoffset} in the \texttt{obsinf} table).
\item[\tt cdiff] gives the difference from ``expected'' column number, i.e.
where it actually is in the image minus what it is expected to be (this is ``on top''
of \texttt{rowoffset} in the \texttt{obsinf} table).
\item[\tt xoffstd] gives the standard deviation in the calculation of
\texttt{cdiff}.
\item[\tt yoffstd] gives the standard deviation in the calculation of
\texttt{rdiff}.
\item[\tt amp] gives the calculated flux of the object (at the last
calculation).
\item[\tt sigma] gives the calculated standard deviation of the flux of the
object (at the last calculation).
\item[\tt ampstd] gives the standard deviation on the calculation of
\texttt{amp}.
\item[\tt sigmastd] gives the standard deviation on the calculation of
\texttt{sigma}.
\item[\tt apsize] gives the aperture size used.
\item[\tt adus] gives the ADU count.
\item[\tt modadus] gives the ADU count taken from the model in use.
\item[\tt hide] indicates that thisi particular entry should be ignored for
whatever reason.
\end{description}

\subsubsection{\texttt{aducalc} table}
\protect\label{section:aducalc}

The following fields are provided in the \texttt{aducalc} table. This records an
actual flux calculation.

\begin{description}
\item[\tt objind] gives the object index.
\item[\tt obsind] gives the observation index.
\item[\tt frind] gives the index in the \texttt{findresult} table.
\item[\tt calcdate] gives the date on which the value was calculated.
\item[\tt skylevel] gives the calculated sky level.
\item[\tt skystd] gives the sky level standard deviation.
\item[\tt apsize] gives the aperture used.
\item[\tt aducount] gives the ADU count recorded.
\item[\tt aduerr] gives the standard deviation of the ADU count.
\item[\tt modaducount] gives the ADU count from the model used.
\item[\tt modaduerr] gives the standard deviation of the ADU count from the
model.
\end{description}
