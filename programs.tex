\section{Main programs}
\protect\label{section:mainprogs}

In this part is documented the programs more or less in their final form. Some
of the original programs were written in Perl but all of the later programs are
written in \py.

The graphical programs are written using Qt4 and later Qt5.\footnote{The extra
features of Qt5 are not used anywhere, but the {\py} \texttt{import} statements
are different.} This currently is offered under a licensing agreement allowing
non-commercial use, but this will have to be installed if it is intended to
develop the software further.

The ``command line'' programs written in {\py} use the \texttt{argparse} library
for decoding arguments. Wherever possible the same arguments with the same
syntax are used and library modules are used to provide these.

\subsection{A note on database usage}
\protect\label{section:databaseusage}

There are 4 databases in use by the software, the credentials of which are all
given appropriately in \texttt{\~/lib/dbcred.ini} as described in section
\ref{section:mysqlpw}.

\begin{description}
\item[\tt remfits] is the name of the database on my home machine, possibly via
remote access to my home machine).
\item[\tt cluster] is the name of the database on the cluster, possibly via an
SSH tunnel.
\item[\tt rdots] is the name of the database on the Red Dots project holding the
observation entries.
\item[\tt rdotsquery] is the name of the database on the Red Dots project
holding the flat and bias entries.
\end{description}

All of the programs that access the database use a common interface (which is
provided for in the module \texttt{remdefaults}).

Using the \texttt{argparse} module, each module uses the database, includes the
following:

\begin{verbatim}
import argparse
import remdefaults

....

parsearg = argparse.ArgumentParser(description='Program description',
            formatter_class=argparse.ArgumentDefaultsHelpFormatter)

remdefaults.parseargs(parsearg)
.... (add other arguments)

resargs = vars(parsearg.parse_args())
remdefaults.getargs(resargs)
\end{verbatim}

The function of this is to insert the following into the help message produced
by running the program with the \texttt{-h} option.

\begin{verbatim}
Program description

options:
  -h, --help            show this help message and exit
  --database DATABASE   Database to use (default: remfits)
\end{verbatim}

The default name (in this instance \texttt{remfits} is obtained from:

\begin{enumerate}
  \item The environment variable \texttt{REMDB} if it exists.
  \item Otherwise the ``hostname'' is examined and \texttt{remfits} is selected
  it is my machine, otherwise \texttt{cluster} is selected.
\end{enumerate}

This means that if you want to explicitly select the database you can provide it
with the option \texttt{\dmin database xyz}, or you can set it up as an environment
variable \texttt{REMDB}, the former taking priority.

The logic for this is all contained in the \texttt{remdefaults} module which
should be amended if necessary to adjust the defaults.

\subsection{Command line programs}
\protect\label{section:commandline}

``Command line'' programs also include those invoked in ``batch'' style using
``cron'' for a nightly run.

As mentioned above, the {\py} command line programs use the \texttt{argparse}
library. This has the main benefit that documentation is more or less automatic
and all programs take the \texttt{-h} option (or \texttt{\dmin help}) to display
all the other options and exit.

\bomb There is a bug in the library, in that some options which take a string
which can be a number in some cases (particularly a range) go wrong with negative
numbers, for example if an argument is a range of the form
\textit{min:max} and the \textit{min} value can be negative, then you might want
the argument to be something like  \texttt{-10:20} and the library may be
confused by the leading minus sign. To overcome this, range type arguments allow
a single letter to precede the minus sign thus \texttt{m-10:20}. This does not
happen often, but there are cases where this crops up.

Perl programs use perl library routines which do not have this bug, however
their options are much more limited.

Standard arguments which apply to nearly all routines are as follows:

\begin{verbatim}
  -h, --help           show this help message and exit
  --database DATABASE  Database to use (default: remfits)
  --logfile LOGFILE    File for log output if not stderr (default: None)
  --verbose            Print out summary of what has been done (default: ....)
  --debug              Debug queries (default: False)
\end{verbatim}

As explained above, the default database depends upon which computer the program
is run on or whether the environment variable \texttt{REMDB} is set.

The log file is where error messages (and commentary given by
\texttt{\dmin verbose}) is displayed. It might be required to send this to a file.

Some of the database queries are complicated and many programs provide a
\texttt{\dmin debug} option to print them out before doing the operation.

\subsubsection{Daily CRON routine}
\protect\label{section:cronroutine}

I use the following as a daily routine to update the database. This is done in
mid-afternoon (to avoid the time when the Red Dots database is updated).

\begin{verbatim}
#! /bin/bash

ANACONDA=/usr/local/anaconda3
ANACONDALIB=$ANACONDA/lib/python3.8
ANACPP=$ANACONDALIB:$ANACONDALIB/site-packages:$PYTHONPATH

PERLLIB=/home/jmc/lib/perllib
PYTHONPATH=/home/jmc/lib/pythonlib
ASTRO=/home/jmc/src/astro
PATH=$ASTRO/progs/Numpy/remfits:$ASTRO/progs/Numpy/remfits2:/home/jmc/bin:$ASTRO/bin:$PATH
REMTMP=/home/jmc/src/astro/tmp

export PERLLIB PYTHONPATH PATH REMTMP

if ! copy_new_rows.py
then
	echo Nothing doing
	exit 0
fi

load_new_obs.py
load_new_iforbs.py
get_new_fits_params.py --trim 90

bary_dates.py
lexi:

\end{verbatim}

After appropriately setting up some environment variables for the various
software and ensuring the \texttt{\$PATH} includes the binaries, certain items
of software are then run to load new rows of observations, bias and flat files.
These programs are all described below.

Note that if nothing has been added to the Red Dots database, the first run
program \texttt{copy\_new\_rows.py} will return a non-zero exit code so that the
script can report \texttt{Nothing doing} and exit.

In the following sections the functions of these programs are given.

\subsubsubsection{Copy\_new\_rows}
\protect\label{section:copynewrows}

The program \texttt{copy\_new\_rows.py} identifies any newly added rows in the
Red Dots databases and copies them to the in-house database, specifically the
tables \texttt{obsinf} and \texttt{iforbinf}.

The assumption made is like the \texttt{ind} fields in the database, the field
\texttt{serial} fields in the two Red Dots databases for observations and the
individual flat and bias files are monotonically increasing. So by searching for
the maximum values of \texttt{serial} held, any rows with fields greater than
this new rows (observations, flat or bias files) cam be identified.

\bomb Note that if this assumption is incorrect it is possible that new rows
will be missed.

The program is normally intended to run without options as in the nightly run
example in the previous section \ref{section:cronroutine}.

Options to \texttt{copy\_new\_rows.py} are as follows:

\begin{description}
\item[\tt \dmin help] display options and exit.
\item[\tt \dmin database] select in-house database to use, default is according
to computer run on.
\item[\tt \dmin logfile \textit{file}] specify a file to output messages to if
not standard error.
\item[\tt \dmin verbose] This option indicates how many rows, i.e. observations and
flat/bias files have been loaded. (Not including the FITS files.). This is on by
default and specifying it turns it off.
\item[\tt \dmin debug] Prints out database queries on standard error or whatever
is specified by \texttt{\dmin logfile}.
\end{description}

The program returns a non-zero exit code if no rows are loaded, which is used in
the ``cron'' routine.

\subsubsubsection{Load\_new\_obs}
\protect\label{section:loadnewobs}

The program \texttt{load\_new\_obs.py} loads any new FITS files corresponding to
the new rows loaded by \texttt{copy\_new\_rows.py}.

The program is normally intended to run without options as in the nightly run
example in the earlier section \ref{section:cronroutine}.

Options to \texttt{load\_new\_obs.py} are as follows:

\begin{description}
\item[\tt \dmin help] display options and exit.
\item[\tt \dmin database] select in-house database to use, default is according
to computer run on.
\item[\tt \dmin logfile \textit{file}] specify a file to output messages to if
not standard error.
\item[\tt \dmin nodates] Do not restrict loaded files by date.
\item[\tt \dmin dates \textit{range}] Specify dates to load files for, this can
be given as \textit{start:end} thus \texttt{1/1/2017:31/12/2022}. The start or
end date may be omitted to omit the limitation to dates after or before a date.
The default if no date s given is \texttt{1/1/2017}.
\item[\tt \dmin allmonth \textit{month}] is an alternative way of specifying the
dates as \textit{year-month} for example \texttt{2019-9} for September 2019.
This selects all dates in the given month.
\item[\tt \dmin remir] indicates that REMIR files should be loaded.
\item[\tt \dmin verbose] gives a summary of what has been loaded (on by
default).
\item[\tt \dmin targets] indicates that files for the 3 Red Dots targets \prox,
{\bstar} and {\ross} should be loaded. On by default.
\item[\tt \dmin objects \textit{objname \ldots}] gives one or more objects in
addition to the targets, if requested.
\item[\tt \dmin debug] Prints out database queries on standard error or whatever
is specified by \texttt{\dmin logfile}.
\end{description}

Note that only one of \texttt{\dmin nodates}, \texttt{\dmin dates} and
\texttt{\dmin allmonth} should be given.

Also note the logic of \texttt{\dmin targets} and \texttt{\dmin objects}.
Selection of the targets is on by default, so that \texttt{\dmin objects~xyz}
will load files for the targets and for \texttt{xyz}. If you only want files for
\texttt{xyz} you should specify \texttt{\dmin targets} as well, to turn off
selection of targets.
